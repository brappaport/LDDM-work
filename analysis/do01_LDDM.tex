% Options for packages loaded elsewhere
\PassOptionsToPackage{unicode}{hyperref}
\PassOptionsToPackage{hyphens}{url}
%
\documentclass[
]{article}
\usepackage{amsmath,amssymb}
\usepackage{lmodern}
\usepackage{iftex}
\ifPDFTeX
  \usepackage[T1]{fontenc}
  \usepackage[utf8]{inputenc}
  \usepackage{textcomp} % provide euro and other symbols
\else % if luatex or xetex
  \usepackage{unicode-math}
  \defaultfontfeatures{Scale=MatchLowercase}
  \defaultfontfeatures[\rmfamily]{Ligatures=TeX,Scale=1}
\fi
% Use upquote if available, for straight quotes in verbatim environments
\IfFileExists{upquote.sty}{\usepackage{upquote}}{}
\IfFileExists{microtype.sty}{% use microtype if available
  \usepackage[]{microtype}
  \UseMicrotypeSet[protrusion]{basicmath} % disable protrusion for tt fonts
}{}
\makeatletter
\@ifundefined{KOMAClassName}{% if non-KOMA class
  \IfFileExists{parskip.sty}{%
    \usepackage{parskip}
  }{% else
    \setlength{\parindent}{0pt}
    \setlength{\parskip}{6pt plus 2pt minus 1pt}}
}{% if KOMA class
  \KOMAoptions{parskip=half}}
\makeatother
\usepackage{xcolor}
\usepackage[margin=1in]{geometry}
\usepackage{color}
\usepackage{fancyvrb}
\newcommand{\VerbBar}{|}
\newcommand{\VERB}{\Verb[commandchars=\\\{\}]}
\DefineVerbatimEnvironment{Highlighting}{Verbatim}{commandchars=\\\{\}}
% Add ',fontsize=\small' for more characters per line
\usepackage{framed}
\definecolor{shadecolor}{RGB}{248,248,248}
\newenvironment{Shaded}{\begin{snugshade}}{\end{snugshade}}
\newcommand{\AlertTok}[1]{\textcolor[rgb]{0.94,0.16,0.16}{#1}}
\newcommand{\AnnotationTok}[1]{\textcolor[rgb]{0.56,0.35,0.01}{\textbf{\textit{#1}}}}
\newcommand{\AttributeTok}[1]{\textcolor[rgb]{0.77,0.63,0.00}{#1}}
\newcommand{\BaseNTok}[1]{\textcolor[rgb]{0.00,0.00,0.81}{#1}}
\newcommand{\BuiltInTok}[1]{#1}
\newcommand{\CharTok}[1]{\textcolor[rgb]{0.31,0.60,0.02}{#1}}
\newcommand{\CommentTok}[1]{\textcolor[rgb]{0.56,0.35,0.01}{\textit{#1}}}
\newcommand{\CommentVarTok}[1]{\textcolor[rgb]{0.56,0.35,0.01}{\textbf{\textit{#1}}}}
\newcommand{\ConstantTok}[1]{\textcolor[rgb]{0.00,0.00,0.00}{#1}}
\newcommand{\ControlFlowTok}[1]{\textcolor[rgb]{0.13,0.29,0.53}{\textbf{#1}}}
\newcommand{\DataTypeTok}[1]{\textcolor[rgb]{0.13,0.29,0.53}{#1}}
\newcommand{\DecValTok}[1]{\textcolor[rgb]{0.00,0.00,0.81}{#1}}
\newcommand{\DocumentationTok}[1]{\textcolor[rgb]{0.56,0.35,0.01}{\textbf{\textit{#1}}}}
\newcommand{\ErrorTok}[1]{\textcolor[rgb]{0.64,0.00,0.00}{\textbf{#1}}}
\newcommand{\ExtensionTok}[1]{#1}
\newcommand{\FloatTok}[1]{\textcolor[rgb]{0.00,0.00,0.81}{#1}}
\newcommand{\FunctionTok}[1]{\textcolor[rgb]{0.00,0.00,0.00}{#1}}
\newcommand{\ImportTok}[1]{#1}
\newcommand{\InformationTok}[1]{\textcolor[rgb]{0.56,0.35,0.01}{\textbf{\textit{#1}}}}
\newcommand{\KeywordTok}[1]{\textcolor[rgb]{0.13,0.29,0.53}{\textbf{#1}}}
\newcommand{\NormalTok}[1]{#1}
\newcommand{\OperatorTok}[1]{\textcolor[rgb]{0.81,0.36,0.00}{\textbf{#1}}}
\newcommand{\OtherTok}[1]{\textcolor[rgb]{0.56,0.35,0.01}{#1}}
\newcommand{\PreprocessorTok}[1]{\textcolor[rgb]{0.56,0.35,0.01}{\textit{#1}}}
\newcommand{\RegionMarkerTok}[1]{#1}
\newcommand{\SpecialCharTok}[1]{\textcolor[rgb]{0.00,0.00,0.00}{#1}}
\newcommand{\SpecialStringTok}[1]{\textcolor[rgb]{0.31,0.60,0.02}{#1}}
\newcommand{\StringTok}[1]{\textcolor[rgb]{0.31,0.60,0.02}{#1}}
\newcommand{\VariableTok}[1]{\textcolor[rgb]{0.00,0.00,0.00}{#1}}
\newcommand{\VerbatimStringTok}[1]{\textcolor[rgb]{0.31,0.60,0.02}{#1}}
\newcommand{\WarningTok}[1]{\textcolor[rgb]{0.56,0.35,0.01}{\textbf{\textit{#1}}}}
\usepackage{graphicx}
\makeatletter
\def\maxwidth{\ifdim\Gin@nat@width>\linewidth\linewidth\else\Gin@nat@width\fi}
\def\maxheight{\ifdim\Gin@nat@height>\textheight\textheight\else\Gin@nat@height\fi}
\makeatother
% Scale images if necessary, so that they will not overflow the page
% margins by default, and it is still possible to overwrite the defaults
% using explicit options in \includegraphics[width, height, ...]{}
\setkeys{Gin}{width=\maxwidth,height=\maxheight,keepaspectratio}
% Set default figure placement to htbp
\makeatletter
\def\fps@figure{htbp}
\makeatother
\setlength{\emergencystretch}{3em} % prevent overfull lines
\providecommand{\tightlist}{%
  \setlength{\itemsep}{0pt}\setlength{\parskip}{0pt}}
\setcounter{secnumdepth}{-\maxdimen} % remove section numbering
\ifLuaTeX
  \usepackage{selnolig}  % disable illegal ligatures
\fi
\IfFileExists{bookmark.sty}{\usepackage{bookmark}}{\usepackage{hyperref}}
\IfFileExists{xurl.sty}{\usepackage{xurl}}{} % add URL line breaks if available
\urlstyle{same} % disable monospaced font for URLs
\hypersetup{
  pdftitle={Data analysis script 1: Preliminary ERP to DDM concordance},
  pdfauthor={Brent Rappaport},
  hidelinks,
  pdfcreator={LaTeX via pandoc}}

\title{Data analysis script 1: Preliminary ERP to DDM concordance}
\usepackage{etoolbox}
\makeatletter
\providecommand{\subtitle}[1]{% add subtitle to \maketitle
  \apptocmd{\@title}{\par {\large #1 \par}}{}{}
}
\makeatother
\subtitle{Template Rmd}
\author{Brent Rappaport}
\date{2022-10-18}

\begin{document}
\maketitle

{
\setcounter{tocdepth}{3}
\tableofcontents
}
\hypertarget{about}{%
\section{About}\label{about}}

This script imports and merges the self-report data from the Flanker
task of the State-Trait study

\hypertarget{get-setup}{%
\section{1. Get Setup}\label{get-setup}}

\hypertarget{clear-everything-set-width}{%
\subsection{1.1. Clear everything \& set
width}\label{clear-everything-set-width}}

\begin{Shaded}
\begin{Highlighting}[]
    \FunctionTok{options}\NormalTok{(}\AttributeTok{width=}\DecValTok{80}\NormalTok{, }\AttributeTok{Ncpus =} \DecValTok{6}\NormalTok{) }\CommentTok{\#Set width}
    \FunctionTok{rm}\NormalTok{(}\AttributeTok{list=}\FunctionTok{ls}\NormalTok{())     }\CommentTok{\#Remove everything from environment}
    \FunctionTok{cat}\NormalTok{(}\StringTok{"}\SpecialCharTok{\textbackslash{}014}\StringTok{"}\NormalTok{)       }\CommentTok{\#Clear Console}
\end{Highlighting}
\end{Shaded}

\hypertarget{load-libraries}{%
\subsection{1.2. Load Libraries}\label{load-libraries}}

\begin{Shaded}
\begin{Highlighting}[]
\NormalTok{  renv}\SpecialCharTok{::}\FunctionTok{restore}\NormalTok{()     }\CommentTok{\#restore environment}
  \FunctionTok{library}\NormalTok{(knitr)      }\CommentTok{\#allows rmarkdown files}
  \FunctionTok{library}\NormalTok{(haven)      }\CommentTok{\#helps import stata}
  \FunctionTok{library}\NormalTok{(questionr)  }\CommentTok{\#allows lookfor function}
  \FunctionTok{library}\NormalTok{(MASS)       }\CommentTok{\#calculate residualized scores}
  \FunctionTok{library}\NormalTok{(tidyverse)  }\CommentTok{\#plotting/cleaning, etc.}
  \FunctionTok{library}\NormalTok{(broom)      }\CommentTok{\#nice statistical output}
  \FunctionTok{library}\NormalTok{(here)       }\CommentTok{\#nice file paths}
  \FunctionTok{library}\NormalTok{(expss)      }\CommentTok{\#labeling variables/values}
  \FunctionTok{library}\NormalTok{(psych)      }\CommentTok{\#used for statistical analyses}
  \FunctionTok{library}\NormalTok{(workflowr)  }\CommentTok{\#helps with workflow}
\end{Highlighting}
\end{Shaded}

\hypertarget{get-the-working-directory}{%
\subsection{1.3. Get the Working
Directory}\label{get-the-working-directory}}

\begin{Shaded}
\begin{Highlighting}[]
  \FunctionTok{here}\NormalTok{()}
\end{Highlighting}
\end{Shaded}

\begin{verbatim}
## [1] "/Volumes/fsmresfiles/PBS/Stewarts/State Trait Study/current_studies/Letkiewicz_DDM/work"
\end{verbatim}

\hypertarget{set-seed}{%
\subsection{1.4. Set seed}\label{set-seed}}

\begin{Shaded}
\begin{Highlighting}[]
     \FunctionTok{set.seed}\NormalTok{(}\DecValTok{312}\NormalTok{)    }\CommentTok{\#Set seed}
\end{Highlighting}
\end{Shaded}

\hypertarget{load-data}{%
\subsection{1.5 Load Data}\label{load-data}}

Remember to immediately rename and remove. Avoid overwriting old data.

\begin{Shaded}
\begin{Highlighting}[]
\FunctionTok{load}\NormalTok{(}\AttributeTok{file=}\FunctionTok{here}\NormalTok{(}\StringTok{"./data/LDDM\_cleaning03.RData"}\NormalTok{))}
\NormalTok{LDDM\_do1 }\OtherTok{\textless{}{-}}\NormalTok{ LDDM\_cleaning03; }\FunctionTok{rm}\NormalTok{(LDDM\_cleaning03)}

\FunctionTok{load}\NormalTok{(}\AttributeTok{file=}\FunctionTok{here}\NormalTok{(}\StringTok{"./data/LDDM\_cleaning03\_d1.RData"}\NormalTok{))}
\NormalTok{LDDM\_do1\_d1 }\OtherTok{\textless{}{-}}\NormalTok{ LDDM\_cleaning03\_d1; }\FunctionTok{rm}\NormalTok{(LDDM\_cleaning03\_d1)}

\FunctionTok{load}\NormalTok{(}\AttributeTok{file=}\FunctionTok{here}\NormalTok{(}\StringTok{"./data/LDDM\_cleaning03\_d2.RData"}\NormalTok{))}
\NormalTok{LDDM\_do1\_d2 }\OtherTok{\textless{}{-}}\NormalTok{ LDDM\_cleaning03\_d2; }\FunctionTok{rm}\NormalTok{(LDDM\_cleaning03\_d2)}
\end{Highlighting}
\end{Shaded}

\hypertarget{correlation-between-erp-measures-and-ddm-parameters}{%
\section{2. Correlation between ERP measures and DDM
parameters}\label{correlation-between-erp-measures-and-ddm-parameters}}

Threshold = z Drift rate = v Decision boundary separation = a
Non-decision time = t

Per Allie, larger values of v mean faster accumulation of evidence
(faster rate), larger t means slower non-decision time processing,
larger a means more separation between the correct and incorrect
response boundaries, and z is the distance from the upper boundary to
the start of the drift process.

\hypertarget{ern-mean-amplitude}{%
\subsection{2.1 ERN mean amplitude}\label{ern-mean-amplitude}}

\begin{Shaded}
\begin{Highlighting}[]
\FunctionTok{corr.test}\NormalTok{(LDDM\_do1\_d1}\SpecialCharTok{$}\NormalTok{FCZ\_ERN\_080, LDDM\_do1\_d1}\SpecialCharTok{$}\NormalTok{z\_S1\_B11, }\AttributeTok{use=}\StringTok{"complete.obs"}\NormalTok{, }\AttributeTok{method=}\StringTok{"pearson"}\NormalTok{)}
\end{Highlighting}
\end{Shaded}

\begin{verbatim}
## Call:corr.test(x = LDDM_do1_d1$FCZ_ERN_080, y = LDDM_do1_d1$z_S1_B11, 
##     use = "complete.obs", method = "pearson")
## Correlation matrix 
## [1] 0.07
## Sample Size 
## [1] 79
## These are the unadjusted probability values.
##   The probability values  adjusted for multiple tests are in the p.adj object. 
## [1] 0.53
## 
##  To see confidence intervals of the correlations, print with the short=FALSE option
\end{verbatim}

\begin{Shaded}
\begin{Highlighting}[]
\FunctionTok{corr.test}\NormalTok{(LDDM\_do1\_d1}\SpecialCharTok{$}\NormalTok{FCZ\_ERN\_0100, LDDM\_do1\_d1}\SpecialCharTok{$}\NormalTok{z\_S1\_B11, }\AttributeTok{use=}\StringTok{"complete.obs"}\NormalTok{, }\AttributeTok{method=}\StringTok{"pearson"}\NormalTok{)}
\end{Highlighting}
\end{Shaded}

\begin{verbatim}
## Call:corr.test(x = LDDM_do1_d1$FCZ_ERN_0100, y = LDDM_do1_d1$z_S1_B11, 
##     use = "complete.obs", method = "pearson")
## Correlation matrix 
## [1] 0.05
## Sample Size 
## [1] 79
## These are the unadjusted probability values.
##   The probability values  adjusted for multiple tests are in the p.adj object. 
## [1] 0.65
## 
##  To see confidence intervals of the correlations, print with the short=FALSE option
\end{verbatim}

\begin{Shaded}
\begin{Highlighting}[]
\FunctionTok{corr.test}\NormalTok{(LDDM\_do1\_d1}\SpecialCharTok{$}\NormalTok{FCZ\_ERN\_080, LDDM\_do1\_d1}\SpecialCharTok{$}\NormalTok{v\_congruent\_S1\_B11, }\AttributeTok{use=}\StringTok{"complete.obs"}\NormalTok{, }\AttributeTok{method=}\StringTok{"pearson"}\NormalTok{)}
\end{Highlighting}
\end{Shaded}

\begin{verbatim}
## Call:corr.test(x = LDDM_do1_d1$FCZ_ERN_080, y = LDDM_do1_d1$v_congruent_S1_B11, 
##     use = "complete.obs", method = "pearson")
## Correlation matrix 
## [1] -0.33
## Sample Size 
## [1] 79
## These are the unadjusted probability values.
##   The probability values  adjusted for multiple tests are in the p.adj object. 
## [1] 0
## 
##  To see confidence intervals of the correlations, print with the short=FALSE option
\end{verbatim}

\begin{Shaded}
\begin{Highlighting}[]
\FunctionTok{corr.test}\NormalTok{(LDDM\_do1\_d1}\SpecialCharTok{$}\NormalTok{FCZ\_ERN\_080, LDDM\_do1\_d1}\SpecialCharTok{$}\NormalTok{v\_incongruent\_S1\_B11, }\AttributeTok{use=}\StringTok{"complete.obs"}\NormalTok{, }\AttributeTok{method=}\StringTok{"pearson"}\NormalTok{)}
\end{Highlighting}
\end{Shaded}

\begin{verbatim}
## Call:corr.test(x = LDDM_do1_d1$FCZ_ERN_080, y = LDDM_do1_d1$v_incongruent_S1_B11, 
##     use = "complete.obs", method = "pearson")
## Correlation matrix 
## [1] -0.33
## Sample Size 
## [1] 79
## These are the unadjusted probability values.
##   The probability values  adjusted for multiple tests are in the p.adj object. 
## [1] 0
## 
##  To see confidence intervals of the correlations, print with the short=FALSE option
\end{verbatim}

\begin{Shaded}
\begin{Highlighting}[]
\FunctionTok{corr.test}\NormalTok{(LDDM\_do1\_d1}\SpecialCharTok{$}\NormalTok{FCZ\_ERN\_0100, LDDM\_do1\_d1}\SpecialCharTok{$}\NormalTok{v\_congruent\_S1\_B11, }\AttributeTok{use=}\StringTok{"complete.obs"}\NormalTok{, }\AttributeTok{method=}\StringTok{"pearson"}\NormalTok{)}
\end{Highlighting}
\end{Shaded}

\begin{verbatim}
## Call:corr.test(x = LDDM_do1_d1$FCZ_ERN_0100, y = LDDM_do1_d1$v_congruent_S1_B11, 
##     use = "complete.obs", method = "pearson")
## Correlation matrix 
## [1] -0.29
## Sample Size 
## [1] 79
## These are the unadjusted probability values.
##   The probability values  adjusted for multiple tests are in the p.adj object. 
## [1] 0.01
## 
##  To see confidence intervals of the correlations, print with the short=FALSE option
\end{verbatim}

\begin{Shaded}
\begin{Highlighting}[]
\FunctionTok{corr.test}\NormalTok{(LDDM\_do1\_d1}\SpecialCharTok{$}\NormalTok{FCZ\_ERN\_0100, LDDM\_do1\_d1}\SpecialCharTok{$}\NormalTok{v\_incongruent\_S1\_B11, }\AttributeTok{use=}\StringTok{"complete.obs"}\NormalTok{, }\AttributeTok{method=}\StringTok{"pearson"}\NormalTok{)}
\end{Highlighting}
\end{Shaded}

\begin{verbatim}
## Call:corr.test(x = LDDM_do1_d1$FCZ_ERN_0100, y = LDDM_do1_d1$v_incongruent_S1_B11, 
##     use = "complete.obs", method = "pearson")
## Correlation matrix 
## [1] -0.27
## Sample Size 
## [1] 79
## These are the unadjusted probability values.
##   The probability values  adjusted for multiple tests are in the p.adj object. 
## [1] 0.01
## 
##  To see confidence intervals of the correlations, print with the short=FALSE option
\end{verbatim}

\begin{Shaded}
\begin{Highlighting}[]
\FunctionTok{corr.test}\NormalTok{(LDDM\_do1\_d1}\SpecialCharTok{$}\NormalTok{FCZ\_ERN\_080, LDDM\_do1\_d1}\SpecialCharTok{$}\NormalTok{a\_S1\_B11, }\AttributeTok{use=}\StringTok{"complete.obs"}\NormalTok{, }\AttributeTok{method=}\StringTok{"pearson"}\NormalTok{)}
\end{Highlighting}
\end{Shaded}

\begin{verbatim}
## Call:corr.test(x = LDDM_do1_d1$FCZ_ERN_080, y = LDDM_do1_d1$a_S1_B11, 
##     use = "complete.obs", method = "pearson")
## Correlation matrix 
## [1] 0.06
## Sample Size 
## [1] 79
## These are the unadjusted probability values.
##   The probability values  adjusted for multiple tests are in the p.adj object. 
## [1] 0.62
## 
##  To see confidence intervals of the correlations, print with the short=FALSE option
\end{verbatim}

\begin{Shaded}
\begin{Highlighting}[]
\FunctionTok{corr.test}\NormalTok{(LDDM\_do1\_d1}\SpecialCharTok{$}\NormalTok{FCZ\_ERN\_0100, LDDM\_do1\_d1}\SpecialCharTok{$}\NormalTok{a\_S1\_B11, }\AttributeTok{use=}\StringTok{"complete.obs"}\NormalTok{, }\AttributeTok{method=}\StringTok{"pearson"}\NormalTok{)}
\end{Highlighting}
\end{Shaded}

\begin{verbatim}
## Call:corr.test(x = LDDM_do1_d1$FCZ_ERN_0100, y = LDDM_do1_d1$a_S1_B11, 
##     use = "complete.obs", method = "pearson")
## Correlation matrix 
## [1] 0.02
## Sample Size 
## [1] 79
## These are the unadjusted probability values.
##   The probability values  adjusted for multiple tests are in the p.adj object. 
## [1] 0.83
## 
##  To see confidence intervals of the correlations, print with the short=FALSE option
\end{verbatim}

\begin{Shaded}
\begin{Highlighting}[]
\FunctionTok{corr.test}\NormalTok{(LDDM\_do1\_d1}\SpecialCharTok{$}\NormalTok{FCZ\_ERN\_080, LDDM\_do1\_d1}\SpecialCharTok{$}\NormalTok{t\_S1\_B11, }\AttributeTok{use=}\StringTok{"complete.obs"}\NormalTok{, }\AttributeTok{method=}\StringTok{"pearson"}\NormalTok{)}
\end{Highlighting}
\end{Shaded}

\begin{verbatim}
## Call:corr.test(x = LDDM_do1_d1$FCZ_ERN_080, y = LDDM_do1_d1$t_S1_B11, 
##     use = "complete.obs", method = "pearson")
## Correlation matrix 
## [1] 0.05
## Sample Size 
## [1] 79
## These are the unadjusted probability values.
##   The probability values  adjusted for multiple tests are in the p.adj object. 
## [1] 0.66
## 
##  To see confidence intervals of the correlations, print with the short=FALSE option
\end{verbatim}

\begin{Shaded}
\begin{Highlighting}[]
\FunctionTok{corr.test}\NormalTok{(LDDM\_do1\_d1}\SpecialCharTok{$}\NormalTok{FCZ\_ERN\_0100, LDDM\_do1\_d1}\SpecialCharTok{$}\NormalTok{t\_S1\_B11, }\AttributeTok{use=}\StringTok{"complete.obs"}\NormalTok{, }\AttributeTok{method=}\StringTok{"pearson"}\NormalTok{)}
\end{Highlighting}
\end{Shaded}

\begin{verbatim}
## Call:corr.test(x = LDDM_do1_d1$FCZ_ERN_0100, y = LDDM_do1_d1$t_S1_B11, 
##     use = "complete.obs", method = "pearson")
## Correlation matrix 
## [1] 0.04
## Sample Size 
## [1] 79
## These are the unadjusted probability values.
##   The probability values  adjusted for multiple tests are in the p.adj object. 
## [1] 0.71
## 
##  To see confidence intervals of the correlations, print with the short=FALSE option
\end{verbatim}

\begin{Shaded}
\begin{Highlighting}[]
\FunctionTok{ggplot}\NormalTok{(LDDM\_do1\_d1, }\FunctionTok{aes}\NormalTok{(}\AttributeTok{x=}\NormalTok{FCZ\_ERN\_080, }\AttributeTok{y=}\NormalTok{v\_congruent\_S1\_B11)) }\SpecialCharTok{+}
  \FunctionTok{geom\_point}\NormalTok{() }\SpecialCharTok{+}
  \FunctionTok{stat\_smooth}\NormalTok{(}\AttributeTok{method=}\StringTok{"lm"}\NormalTok{)}
\end{Highlighting}
\end{Shaded}

\begin{verbatim}
## `geom_smooth()` using formula 'y ~ x'
\end{verbatim}

\includegraphics{do01_LDDM_files/figure-latex/unnamed-chunk-6-1.pdf} I
\emph{think} this is saying that the larger the ERN amplitude (since a
more negative ERN is a larger amplitude), the faster the drift
rate/accumulation of evidence (v). This is true for congruent and
incongruent stimuli, and at FCz, Fz, and Cz.

\hypertarget{p3-onset}{%
\subsection{2.2 P3 onset}\label{p3-onset}}

\begin{Shaded}
\begin{Highlighting}[]
\FunctionTok{corr.test}\NormalTok{(LDDM\_do1\_d1}\SpecialCharTok{$}\NormalTok{Congruent\_PZ\_fpl, LDDM\_do1\_d1}\SpecialCharTok{$}\NormalTok{z\_S1\_B11, }\AttributeTok{use=}\StringTok{"complete.obs"}\NormalTok{, }\AttributeTok{method=}\StringTok{"pearson"}\NormalTok{)}
\end{Highlighting}
\end{Shaded}

\begin{verbatim}
## Call:corr.test(x = LDDM_do1_d1$Congruent_PZ_fpl, y = LDDM_do1_d1$z_S1_B11, 
##     use = "complete.obs", method = "pearson")
## Correlation matrix 
## [1] -0.06
## Sample Size 
## [1] 79
## These are the unadjusted probability values.
##   The probability values  adjusted for multiple tests are in the p.adj object. 
## [1] 0.59
## 
##  To see confidence intervals of the correlations, print with the short=FALSE option
\end{verbatim}

\begin{Shaded}
\begin{Highlighting}[]
\FunctionTok{corr.test}\NormalTok{(LDDM\_do1\_d1}\SpecialCharTok{$}\NormalTok{Incongruent\_PZ\_fpl, LDDM\_do1\_d1}\SpecialCharTok{$}\NormalTok{z\_S1\_B11, }\AttributeTok{use=}\StringTok{"complete.obs"}\NormalTok{, }\AttributeTok{method=}\StringTok{"pearson"}\NormalTok{)}
\end{Highlighting}
\end{Shaded}

\begin{verbatim}
## Call:corr.test(x = LDDM_do1_d1$Incongruent_PZ_fpl, y = LDDM_do1_d1$z_S1_B11, 
##     use = "complete.obs", method = "pearson")
## Correlation matrix 
## [1] -0.05
## Sample Size 
## [1] 79
## These are the unadjusted probability values.
##   The probability values  adjusted for multiple tests are in the p.adj object. 
## [1] 0.65
## 
##  To see confidence intervals of the correlations, print with the short=FALSE option
\end{verbatim}

\begin{Shaded}
\begin{Highlighting}[]
\FunctionTok{corr.test}\NormalTok{(LDDM\_do1\_d1}\SpecialCharTok{$}\NormalTok{Congruent\_PZ\_fpl, LDDM\_do1\_d1}\SpecialCharTok{$}\NormalTok{v\_congruent\_S1\_B11, }\AttributeTok{use=}\StringTok{"complete.obs"}\NormalTok{, }\AttributeTok{method=}\StringTok{"pearson"}\NormalTok{)}
\end{Highlighting}
\end{Shaded}

\begin{verbatim}
## Call:corr.test(x = LDDM_do1_d1$Congruent_PZ_fpl, y = LDDM_do1_d1$v_congruent_S1_B11, 
##     use = "complete.obs", method = "pearson")
## Correlation matrix 
## [1] -0.32
## Sample Size 
## [1] 79
## These are the unadjusted probability values.
##   The probability values  adjusted for multiple tests are in the p.adj object. 
## [1] 0
## 
##  To see confidence intervals of the correlations, print with the short=FALSE option
\end{verbatim}

\begin{Shaded}
\begin{Highlighting}[]
\FunctionTok{corr.test}\NormalTok{(LDDM\_do1\_d1}\SpecialCharTok{$}\NormalTok{Congruent\_PZ\_fpl, LDDM\_do1\_d1}\SpecialCharTok{$}\NormalTok{v\_incongruent\_S1\_B11, }\AttributeTok{use=}\StringTok{"complete.obs"}\NormalTok{, }\AttributeTok{method=}\StringTok{"pearson"}\NormalTok{)}
\end{Highlighting}
\end{Shaded}

\begin{verbatim}
## Call:corr.test(x = LDDM_do1_d1$Congruent_PZ_fpl, y = LDDM_do1_d1$v_incongruent_S1_B11, 
##     use = "complete.obs", method = "pearson")
## Correlation matrix 
## [1] 0.08
## Sample Size 
## [1] 79
## These are the unadjusted probability values.
##   The probability values  adjusted for multiple tests are in the p.adj object. 
## [1] 0.5
## 
##  To see confidence intervals of the correlations, print with the short=FALSE option
\end{verbatim}

\begin{Shaded}
\begin{Highlighting}[]
\FunctionTok{corr.test}\NormalTok{(LDDM\_do1\_d1}\SpecialCharTok{$}\NormalTok{Incongruent\_PZ\_fpl, LDDM\_do1\_d1}\SpecialCharTok{$}\NormalTok{v\_congruent\_S1\_B11, }\AttributeTok{use=}\StringTok{"complete.obs"}\NormalTok{, }\AttributeTok{method=}\StringTok{"pearson"}\NormalTok{)}
\end{Highlighting}
\end{Shaded}

\begin{verbatim}
## Call:corr.test(x = LDDM_do1_d1$Incongruent_PZ_fpl, y = LDDM_do1_d1$v_congruent_S1_B11, 
##     use = "complete.obs", method = "pearson")
## Correlation matrix 
## [1] -0.05
## Sample Size 
## [1] 79
## These are the unadjusted probability values.
##   The probability values  adjusted for multiple tests are in the p.adj object. 
## [1] 0.67
## 
##  To see confidence intervals of the correlations, print with the short=FALSE option
\end{verbatim}

\begin{Shaded}
\begin{Highlighting}[]
\FunctionTok{corr.test}\NormalTok{(LDDM\_do1\_d1}\SpecialCharTok{$}\NormalTok{Incongruent\_PZ\_fpl, LDDM\_do1\_d1}\SpecialCharTok{$}\NormalTok{v\_incongruent\_S1\_B11, }\AttributeTok{use=}\StringTok{"complete.obs"}\NormalTok{, }\AttributeTok{method=}\StringTok{"pearson"}\NormalTok{)}
\end{Highlighting}
\end{Shaded}

\begin{verbatim}
## Call:corr.test(x = LDDM_do1_d1$Incongruent_PZ_fpl, y = LDDM_do1_d1$v_incongruent_S1_B11, 
##     use = "complete.obs", method = "pearson")
## Correlation matrix 
## [1] 0.16
## Sample Size 
## [1] 79
## These are the unadjusted probability values.
##   The probability values  adjusted for multiple tests are in the p.adj object. 
## [1] 0.16
## 
##  To see confidence intervals of the correlations, print with the short=FALSE option
\end{verbatim}

\begin{Shaded}
\begin{Highlighting}[]
\FunctionTok{corr.test}\NormalTok{(LDDM\_do1\_d1}\SpecialCharTok{$}\NormalTok{Congruent\_PZ\_fpl, LDDM\_do1\_d1}\SpecialCharTok{$}\NormalTok{a\_S1\_B11, }\AttributeTok{use=}\StringTok{"complete.obs"}\NormalTok{, }\AttributeTok{method=}\StringTok{"pearson"}\NormalTok{)}
\end{Highlighting}
\end{Shaded}

\begin{verbatim}
## Call:corr.test(x = LDDM_do1_d1$Congruent_PZ_fpl, y = LDDM_do1_d1$a_S1_B11, 
##     use = "complete.obs", method = "pearson")
## Correlation matrix 
## [1] 0.36
## Sample Size 
## [1] 79
## These are the unadjusted probability values.
##   The probability values  adjusted for multiple tests are in the p.adj object. 
## [1] 0
## 
##  To see confidence intervals of the correlations, print with the short=FALSE option
\end{verbatim}

\begin{Shaded}
\begin{Highlighting}[]
\FunctionTok{corr.test}\NormalTok{(LDDM\_do1\_d1}\SpecialCharTok{$}\NormalTok{Incongruent\_PZ\_fpl, LDDM\_do1\_d1}\SpecialCharTok{$}\NormalTok{a\_S1\_B11, }\AttributeTok{use=}\StringTok{"complete.obs"}\NormalTok{, }\AttributeTok{method=}\StringTok{"pearson"}\NormalTok{)}
\end{Highlighting}
\end{Shaded}

\begin{verbatim}
## Call:corr.test(x = LDDM_do1_d1$Incongruent_PZ_fpl, y = LDDM_do1_d1$a_S1_B11, 
##     use = "complete.obs", method = "pearson")
## Correlation matrix 
## [1] 0.29
## Sample Size 
## [1] 79
## These are the unadjusted probability values.
##   The probability values  adjusted for multiple tests are in the p.adj object. 
## [1] 0.01
## 
##  To see confidence intervals of the correlations, print with the short=FALSE option
\end{verbatim}

\begin{Shaded}
\begin{Highlighting}[]
\FunctionTok{corr.test}\NormalTok{(LDDM\_do1\_d1}\SpecialCharTok{$}\NormalTok{Congruent\_PZ\_fpl, LDDM\_do1\_d1}\SpecialCharTok{$}\NormalTok{t\_S1\_B11, }\AttributeTok{use=}\StringTok{"complete.obs"}\NormalTok{, }\AttributeTok{method=}\StringTok{"pearson"}\NormalTok{)}
\end{Highlighting}
\end{Shaded}

\begin{verbatim}
## Call:corr.test(x = LDDM_do1_d1$Congruent_PZ_fpl, y = LDDM_do1_d1$t_S1_B11, 
##     use = "complete.obs", method = "pearson")
## Correlation matrix 
## [1] -0.06
## Sample Size 
## [1] 79
## These are the unadjusted probability values.
##   The probability values  adjusted for multiple tests are in the p.adj object. 
## [1] 0.61
## 
##  To see confidence intervals of the correlations, print with the short=FALSE option
\end{verbatim}

\begin{Shaded}
\begin{Highlighting}[]
\FunctionTok{corr.test}\NormalTok{(LDDM\_do1\_d1}\SpecialCharTok{$}\NormalTok{Incongruent\_PZ\_fpl, LDDM\_do1\_d1}\SpecialCharTok{$}\NormalTok{t\_S1\_B11, }\AttributeTok{use=}\StringTok{"complete.obs"}\NormalTok{, }\AttributeTok{method=}\StringTok{"pearson"}\NormalTok{)}
\end{Highlighting}
\end{Shaded}

\begin{verbatim}
## Call:corr.test(x = LDDM_do1_d1$Incongruent_PZ_fpl, y = LDDM_do1_d1$t_S1_B11, 
##     use = "complete.obs", method = "pearson")
## Correlation matrix 
## [1] 0.08
## Sample Size 
## [1] 79
## These are the unadjusted probability values.
##   The probability values  adjusted for multiple tests are in the p.adj object. 
## [1] 0.47
## 
##  To see confidence intervals of the correlations, print with the short=FALSE option
\end{verbatim}

\begin{Shaded}
\begin{Highlighting}[]
\FunctionTok{ggplot}\NormalTok{(LDDM\_do1\_d1, }\FunctionTok{aes}\NormalTok{(}\AttributeTok{x=}\NormalTok{Congruent\_PZ\_fpl, }\AttributeTok{y=}\NormalTok{v\_congruent\_S1\_B11)) }\SpecialCharTok{+}
  \FunctionTok{geom\_point}\NormalTok{() }\SpecialCharTok{+}
  \FunctionTok{stat\_smooth}\NormalTok{(}\AttributeTok{method=}\StringTok{"lm"}\NormalTok{)}
\end{Highlighting}
\end{Shaded}

\begin{verbatim}
## `geom_smooth()` using formula 'y ~ x'
\end{verbatim}

\includegraphics{do01_LDDM_files/figure-latex/unnamed-chunk-7-1.pdf}

\begin{Shaded}
\begin{Highlighting}[]
\FunctionTok{ggplot}\NormalTok{(LDDM\_do1\_d1, }\FunctionTok{aes}\NormalTok{(}\AttributeTok{x=}\NormalTok{Congruent\_PZ\_fpl, }\AttributeTok{y=}\NormalTok{a\_S1\_B11)) }\SpecialCharTok{+}
  \FunctionTok{geom\_point}\NormalTok{() }\SpecialCharTok{+}
  \FunctionTok{stat\_smooth}\NormalTok{(}\AttributeTok{method=}\StringTok{"lm"}\NormalTok{)}
\end{Highlighting}
\end{Shaded}

\begin{verbatim}
## `geom_smooth()` using formula 'y ~ x'
\end{verbatim}

\includegraphics{do01_LDDM_files/figure-latex/unnamed-chunk-7-2.pdf} 1)
I \emph{think} this is saying that the quicker the P3 onset \emph{to
congruent stimuli} at Pz, the faster the drift rate/accumulation of
evidence (v) \emph{to congruent stimuli}. This is somewhat the case for
\emph{incongruent} but not significant (p = 0.16).

\begin{enumerate}
\def\labelenumi{\arabic{enumi})}
\setcounter{enumi}{1}
\tightlist
\item
  The slower the P3 onset to both congruent and incongruent stimuli at
  Pz, the greater the separation between the two boundaries (a).
\end{enumerate}

No ERP measures were related to non-decision time (t) nor distance from
the upper boundary to the start of the drift process (z).

\begin{Shaded}
\begin{Highlighting}[]
\FunctionTok{library}\NormalTok{(corrplot)}
\end{Highlighting}
\end{Shaded}

\begin{verbatim}
## corrplot 0.92 loaded
\end{verbatim}

\begin{Shaded}
\begin{Highlighting}[]
\NormalTok{all\_measures }\OtherTok{\textless{}{-}} \FunctionTok{c}\NormalTok{(}\StringTok{"FCZ\_ERN\_080"}\NormalTok{,}\StringTok{"FCZ\_ERN\_0100"}\NormalTok{,}\StringTok{"Congruent\_PZ\_fpl"}\NormalTok{,}\StringTok{"Incongruent\_PZ\_fpl"}\NormalTok{,}
                  \StringTok{"z\_S1\_B11"}\NormalTok{,}\StringTok{"v\_congruent\_S1\_B11"}\NormalTok{,}\StringTok{"v\_incongruent\_S1\_B11"}\NormalTok{,}\StringTok{"a\_S1\_B11"}\NormalTok{,}\StringTok{"t\_S1\_B11"}\NormalTok{)}
\FunctionTok{corrplot}\NormalTok{(}\FunctionTok{cor}\NormalTok{(LDDM\_do1\_d1[}\FunctionTok{c}\NormalTok{(all\_measures)],}\AttributeTok{use=}\StringTok{"pairwise.complete.obs"}\NormalTok{), }\AttributeTok{method=}\StringTok{"circle"}\NormalTok{)}
\end{Highlighting}
\end{Shaded}

\includegraphics{do01_LDDM_files/figure-latex/unnamed-chunk-8-1.pdf}

\hypertarget{day-2}{%
\subsection{2.3 Day 2}\label{day-2}}

\begin{Shaded}
\begin{Highlighting}[]
\FunctionTok{corr.test}\NormalTok{(LDDM\_do1\_d2}\SpecialCharTok{$}\NormalTok{FCZ\_ERN\_080, LDDM\_do1\_d2}\SpecialCharTok{$}\NormalTok{v\_congruent\_S2\_B11, }\AttributeTok{use=}\StringTok{"complete.obs"}\NormalTok{, }\AttributeTok{method=}\StringTok{"pearson"}\NormalTok{)}
\FunctionTok{corr.test}\NormalTok{(LDDM\_do1\_d2}\SpecialCharTok{$}\NormalTok{FCZ\_ERN\_080, LDDM\_do1\_d2}\SpecialCharTok{$}\NormalTok{v\_incongruent\_S2\_B11, }\AttributeTok{use=}\StringTok{"complete.obs"}\NormalTok{, }\AttributeTok{method=}\StringTok{"pearson"}\NormalTok{)}
\FunctionTok{corr.test}\NormalTok{(LDDM\_do1\_d2}\SpecialCharTok{$}\NormalTok{FCZ\_ERN\_0100, LDDM\_do1\_d2}\SpecialCharTok{$}\NormalTok{v\_congruent\_S2\_B11, }\AttributeTok{use=}\StringTok{"complete.obs"}\NormalTok{, }\AttributeTok{method=}\StringTok{"pearson"}\NormalTok{)}
\FunctionTok{corr.test}\NormalTok{(LDDM\_do1\_d2}\SpecialCharTok{$}\NormalTok{FCZ\_ERN\_0100, LDDM\_do1\_d2}\SpecialCharTok{$}\NormalTok{v\_incongruent\_S2\_B11, }\AttributeTok{use=}\StringTok{"complete.obs"}\NormalTok{, }\AttributeTok{method=}\StringTok{"pearson"}\NormalTok{)}

\DocumentationTok{\#\#\#\#\#\#\#\#\#\#\#\#\#\#\#\#\#\#\#\#\#\#\#\#\#\#\#\#\#\#\#\#\#\#\#\#\#\#\#\#\#\#\#\#\#\#\#\#\#\#\#\#\#\#\#\#\#\#\#\#\#\#\#\#\#\#\#\#\#\#\#\#\#\#\#\#\#\#\#\#\#\#\#\#\#\#\#\#\#\#\#\#\#\#\#\#\#\#\#\#\#\#\#\#\#}

\FunctionTok{corr.test}\NormalTok{(LDDM\_do1\_d2}\SpecialCharTok{$}\NormalTok{Congruent\_PZ\_fpl, LDDM\_do1\_d2}\SpecialCharTok{$}\NormalTok{v\_congruent\_S2\_B11, }\AttributeTok{use=}\StringTok{"complete.obs"}\NormalTok{, }\AttributeTok{method=}\StringTok{"pearson"}\NormalTok{)}
\FunctionTok{corr.test}\NormalTok{(LDDM\_do1\_d2}\SpecialCharTok{$}\NormalTok{Congruent\_PZ\_fpl, LDDM\_do1\_d2}\SpecialCharTok{$}\NormalTok{v\_incongruent\_S2\_B11, }\AttributeTok{use=}\StringTok{"complete.obs"}\NormalTok{, }\AttributeTok{method=}\StringTok{"pearson"}\NormalTok{)}
\FunctionTok{corr.test}\NormalTok{(LDDM\_do1\_d2}\SpecialCharTok{$}\NormalTok{Incongruent\_PZ\_fpl, LDDM\_do1\_d2}\SpecialCharTok{$}\NormalTok{v\_congruent\_S2\_B11, }\AttributeTok{use=}\StringTok{"complete.obs"}\NormalTok{, }\AttributeTok{method=}\StringTok{"pearson"}\NormalTok{)}
\FunctionTok{corr.test}\NormalTok{(LDDM\_do1\_d2}\SpecialCharTok{$}\NormalTok{Incongruent\_PZ\_fpl, LDDM\_do1\_d2}\SpecialCharTok{$}\NormalTok{v\_incongruent\_S2\_B11, }\AttributeTok{use=}\StringTok{"complete.obs"}\NormalTok{, }\AttributeTok{method=}\StringTok{"pearson"}\NormalTok{)}

\FunctionTok{corr.test}\NormalTok{(LDDM\_do1\_d2}\SpecialCharTok{$}\NormalTok{Congruent\_PZ\_fpl, LDDM\_do1\_d2}\SpecialCharTok{$}\NormalTok{a\_S2\_B11, }\AttributeTok{use=}\StringTok{"complete.obs"}\NormalTok{, }\AttributeTok{method=}\StringTok{"pearson"}\NormalTok{)}
\FunctionTok{corr.test}\NormalTok{(LDDM\_do1\_d2}\SpecialCharTok{$}\NormalTok{Incongruent\_PZ\_fpl, LDDM\_do1\_d2}\SpecialCharTok{$}\NormalTok{a\_S2\_B11, }\AttributeTok{use=}\StringTok{"complete.obs"}\NormalTok{, }\AttributeTok{method=}\StringTok{"pearson"}\NormalTok{)}
\end{Highlighting}
\end{Shaded}

\hypertarget{closing-out}{%
\section{3. Closing out}\label{closing-out}}

In this step, go ahead and close out of the file and quit R without
saving\\
the work space.

\begin{Shaded}
\begin{Highlighting}[]
\NormalTok{   renv}\SpecialCharTok{::}\FunctionTok{snapshot}\NormalTok{()   }\CommentTok{\#Take a snapshot of environment}
\end{Highlighting}
\end{Shaded}

\begin{verbatim}
## * Lockfile written to '/Volumes/fsmresfiles/PBS/Stewarts/State Trait Study/current_studies/Letkiewicz_DDM/work/renv.lock'.
\end{verbatim}

\end{document}
